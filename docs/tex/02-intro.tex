\chapter*{Введение}
\addcontentsline{toc}{chapter}{Введение}

Удаленное подключение к терминалу позволяет пользователю управлять работой удаленного сервера или рабочей станции по сети. Такой подход может быть полезен, например, системным администраторам для администрирования выделенных серверов и технической поддержки пользователей. На основе удаленного подключения реализуется возможность делиться сессией терминала, то есть использовать терминал несколькими пользователями из разных сетей одновременно. С помощью этой возможности можно совместно редактировать файлы, отлаживать какой-либо проект командой разработчиков и так далее.

Целью данной работой является разработка программного обеспечения, позволяющего удаленно подключения терминалу нескольких пользователей и его редактирования. Для достижения поставленной цели необходимо решить следующие задачи:

\begin{itemize}
	\item[---] провести анализ существующих решений;
	\item[---] изучить существующие протоколы прикладного уровня;
	\item[---] разработать свой протокол прикладного уровня для поставленной задачи;
	\item[---] реализовать программное обеспечение (сервер и клиент) с использованием разработанного протокола для удаленного подключения к терминалу.
\end{itemize}
